% \begin{table*}[]
%     \centering
%     \small
%     \begin{tabular}{ll|rr|rr|rr}
%     \toprule
%      &  & \multicolumn{2}{c}{Original eval set} & \multicolumn{2}{c}{Eval examples in train} & \multicolumn{2}{c}{ Unique eval examples} \\
%     Dataset & Model & \multicolumn{1}{c}{Count} & \multicolumn{1}{c}{Ppl} & \multicolumn{1}{c}{Count} & \multicolumn{1}{c}{Ppl} & \multicolumn{1}{c}{Count} & \multicolumn{1}{c}{Ppl} \\
%     \midrule
%     LM1B & Transformer-XL \citep{dai2019transformer} & 306,688 & 21.77  & 15,126  & 10.11  & & \\
%     % RealNews & GROVER-Base &  &  15.440 & & 13.771 \\
%     RealNews & GROVER-Base \citep{zellers2019defending} & 1,639,104 &  15.44 & 235,235 & 13.77 &  & 15.73  \\
%     RealNews & Grover-XL \citep{zellers2019defending} & &  &  & 7.68 & & 9.45 \\
%     \bottomrule
%     \end{tabular}
%     \caption{The perplexity of a trained language model on the official validation set and the perplexity on the valid set examples which were identified as near-duplicates of training set examples. \todo{finish up computing values}}
%     \label{tab:ppl_sota_models}
% \end{table*}


\begin{table}[]
    \caption{For each model, the perplexity of the official validation set (\textit{Orig}), valid set examples which were identified by \Approx{} as matches of train set examples (\textit{Dups}), and valid set examples identified by \Approx{} as unique (\textit{Unique}).
    Due to the size of the RealNews validation set, we evaluated on only the first 25k examples meeting each condition.}
    \label{tab:ppl_sota_models}
    \centering
    \small
    \begin{tabular}{ll|rrr}
    \toprule
    Model & Dataset & \multicolumn{1}{c}{Orig} & \multicolumn{1}{c}{Dups} & \multicolumn{1}{c}{Unique} \\
    \midrule
    Transformer-XL & LM1B & 21.77 & 10.11  & 23.58  \\
    % RealNews & GROVER-Base &  &  15.440 & & 13.771 \\
    GROVER-Base  & RealNews & 15.44 & 13.77 & 15.73  \\
    GROVER-XL & RealNews & 9.15 & 7.68 & 9.45 \\
    \bottomrule
    \end{tabular}
\end{table}