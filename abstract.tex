% \chapter*{ABSTRACT}
% \phantomsection
% \addcontentsline{toc}{chapter}{ABSTRACT} % This is to include this section in the Table of Contents
\begin{center}
{\Large \mytitle}

\vspace{2cm}

{\Large\spacedallcaps{ABSTRACT}}
\end{center}
% \vspace{1cm}

State-of-the-art neural language models are capable of generating incredibly fluent English text.
This success provides opportunities for novel forms of interaction, where human writers work collaboratively with a natural-language generation system toward a set of goals.
However, it also poses several challenges.
Evaluating and comparing the skill of different open-ended text generation systems is challenging, and generated text can have negative societal impact if it proliferates and is not detectable by humans.
In this dissertation, I introduce a detection-based evaluation task that can be used to compare different language models and generative configuations.
By both asking humans to complete this task and training automatic classifier to complete it, I investigate how the tradeoff between generating high-quality and generating diverse text impacts detectability.
Through subsequent large-scale user studies, I show that factors such as the model size and the topic of the generation can have significant influence on human detection capabability.
I show how large neural language models' capability of memorizing large swaths of their training data complicates our ability to evaluate their skill at generating high-quality \textit{novel} text.
I also show how, despite these challenges, neural language models can be successfully employed to support creative writing tasks.
In particular, I describe methods for performing style transfer into any user-provided style and for efficiently supporting fill-in-the-blank operations in addition to the more standard continuation operation.
Finally, I introduce an interactive writing tool we built which allows creative writers to collaborate with a natural language generation system to craft stories.
Users studies with both novice and professional writers provide insights into the strengths and limitations of applying natural language generation systems in real-world settings.